In der Astrophysik beschäftigt sich die Intensitätsinterferometrie (II) mit der Korrelation gemessener Intensitäten, um auf Größen wie Sternendurchmesser schließen zu können. 
Im Gegensatz zu Amplitudeninterferometern sind Intensitätsinterferometer unabhängig von Phasenunterschieden zwischen den interferierenden Wellen und sind daher einfacher in Aufbau und Betrieb. 
Die ersten Observationen mit dieser Methode wurden von Hanbury Brown und Twiss in den 1960ern und 70ern mit einem eigens für diesen Zweck entwickelten Teleskop, dem \emph{Narrabi Stellar Intensity Interferometer} durchgeführt \cite{brownIntensityInterferometerIts1974}. 
Nach einer längeren Pause in der Entwicklung der II, stellte sich in den letzten Jahren die Nutzung bereits bestehender Teleskope, nämlich von \emph{Imaging Atmospheric Cherenkov Telecopes}, kurz IACTs, als sinnvolle Alternative heraus. 
Diese weisen eine große Lichtsammelfläche auf und können für ihren Hauptzweck (der Detektion leuchtschwachen Cherenkov-Lichts) während Phasen, in denen der Mond hell ist, nicht genutzt werden \cite{zmijaFirstIntensityInterferometry2023}. 
Beide Faktoren begünstigen eine Doppelnutzung der IACTs zur Detektion von hochenergetischer Gammastrahlung und -- in Nächten, die die Observation dieser verhindern -- zur Intensitätsinterferometrie. 
So werden beispielsweise die MAGIC- und VERITAS-Teleskope bereits in dieser Konfiguration genutzt \cite{acciariOpticalIntensityInterferometry2020,DemonstrationStellarIntensity}. 
Auch die Astro-Quantum-Optics-Gruppe des \emph{Erlangen Center for Astroparticle Physics} (ECAP) beteiligt sich an einer solchen Kollaboration, nämlich mit dem \emph{High Energy Stereoscopic System} (H.E.S.S.) in Namibia. 
So wurden, neben der Entwicklung des interferometrischen Aufbaus und Labortests, bereits zwei Kampagnen durchgeführt, an denen erfolgreich Winkeldurchmesser heller Sterne der Südhalbkugel gemessen wurden \cite{zmijaOpticalIntensityInterferometry2021,zmijaFirstIntensityInterferometry2023}. 
Ein weiteres Projekt neben der Planung zukünftiger Kampagnen besteht in der Entwicklung und den Tests kleiner, portabler II-Teleskope auf Basis von Fresnel-Linsen, dem sog. \emph{Mobile Intensity Interferometer for Stellar Observations} (MI$^2$SO). \\

Als Teil angesprochener Arbeitsgruppe sollen Untersuchungen im Labor auch ein großer Teil der folgenden Arbeit sein. 
So weisen die gemessenen Photonenkorrelationen der H.E.S.S.-Kampagnen eine Kabellängenabhängigkeit auf, welche allerdings nicht erwartet wird. 
Ziel der vorliegenden Arbeit ist daher, diesen Effekt mit statistisch aussagekräftigeren Labormessungen zu untersuchen. 
Im Zuge dessen wird einerseits ausführlich auf die bereits vorhandene und im Rahmen der Arbeit entwickelte Datenverarbeitung und -analyse eingegangen. 
Andererseits werden Limitationen ebendieser aufgezeigt, um als Resultat einen statistisch stichhaltigen Vergleich verschiedener im Labor erhaltener Photonenkorrelationen, ausgedrückt durch die Kohärenzzeit $\tau_\mathrm{c}$, für unterschiedliche Kabellängen zu erhalten. \\
Die Arbeit ist wie folgt gegliedert: 
In \autoref{sec:Theorie} werden nötige Konzepte der Theorie eingeführt und es erfolgt ein Vergleich der Amplitudeninterferometrie mit der Intensitätsinterferometrie. 
Darauffolgend werden in \autoref{sec:Aufbau} der Messaufbau sowie ausgehend von diesem Erwartungen an $\tau_{\mathrm{c}}$ dargestellt, woraufhin in \autoref{sec:Datenaufnahme und Pre-Processing} die Datenaufnahme und das nötige Pre-Processing erläutert werden. 
Als Hauptteil dieser Arbeit folgt dann \autoref{sec:Analyse}, in dem es um weitere entwickelte Analyseschritte sowie die Auswertung der im Labor aufgenommenen Daten und den Vergleich mit aufgenommenen H.E.S.S.-Daten von 2022 geht. 
Abschließend soll in \autoref{sec:Fazit} noch eine Zusammenfassung der Ergebnisse erfolgen und ein Fazit über den Einfluss der Kabellänge auf $\tau_{\mathrm{c}}$ gezogen werden.